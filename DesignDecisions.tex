%Andrew Butler
%Design Decisions 

\documentclass[letterpaper,12pt]{article}


\begin{document}
	\section*{Design Decisions}
	\subsection*{Language Decisions}
For our Goofy Lights Editor, we will be using Java. Java is convenient as it works easily across platforms, is very readable, and provides its own graphics package, the Swing package. Swing is designed to make APIs that work and look basically the same across all platforms that run Java. It's a package built into the language, no external libraries are needed. We considered using the Qt integration for Java, Qt Jambi, but we decided that this would add complexity without providing any real functionality. 
		
	\subsection*{Complexity Decisions}
When planning this project, several decisions were made to limit the scope and complexity of the final product. These decisions are detailed below. 
	\subsubsection*{Constant Positions}
We decided to support only constant positions of each node. If the nodes were free to move around during a performance, and the software needed to support such moving, the complexity would increase drastically. By limiting our editor to only allow a single configuration of nodes, per performance, we should be able to accomplish our goals. 
	\subsubsection*{Grid Scalability}
As stated previously, we will not allow changing of the grid during a performance. However, the grid should be configurable for the performance as a whole. The configuration component of our program will allow for this. The user will be able to set the dimensions of the grid to be used for the performance they are creating. This will mean that, in some cases, our editor will display a grid larger than the current number of nodes. In such a case the active nodes will be indicated. We decided that a solid rectangular grid will be the only supported configuration. The user will have to account for any unused nodes in the rectangle. Limiting the grid to be purely rectangular dramatically reduces the amount of options our editor would need to support, and we figure that a user will probably want a rectangular configuration in most if not all cases.
\pagebreak[4]
	\subsubsection*{Letters and Default Animations}
After viewing the current editor for the Tower of Lights, we decided that it would be convenient to add some default configurations for our editor. These would include static letters, as well as some built in animations such as the letters animating from one end of the grid to another. Adding this functionality should significantly improve ease of use, while not adding insurmountable complexity to our project. 
		
	
	
\end{document}