%%%%%%%%%%%%%%%%%%%%%%%%%%%%%%%%%%%%%%%%%%%%%%%%%%%%%%%%%%%%%
%															%
% %CS_383_Assignment1_LaTeX_UML_Document					%
% LaTeX Template											%
% Version 1.0 (2/9/17)										%
%															%
%															%
% author: Adrian Beehner									%
%															%
%															%
% description: LaTeX document that contains memo and		%
%	appendicies for UML diagramming tools					%
%															%
%%%%%%%%%%%%%%%%%%%%%%%%%%%%%%%%%%%%%%%%%%%%%%%%%%%%%%%%%%%%%


% Set Up Document
\documentclass[12pt]{article}
\usepackage{titlesec} 
\usepackage{hyperref}
\usepackage{graphicx}
\usepackage{float}
\usepackage{parskip} % Adds spacing between paragraphs
\usepackage{xcolor} % For setting colors
\usepackage{listings}
\setlength{\parindent}{15pt} % Indent paragraphs
\usepackage[margin=1in]{geometry}

% Make so Picture/floats go at top of page
\makeatletter
\setlength{\@fptop}{0pt}
\makeatother

% set the default code style
\lstset
{
	frame=tb, % draw a frame at the top and bottom of the code block
	tabsize=3, % tab space width
	showstringspaces=false, % don't mark spaces in strings
	numbers=left, % display line numbers on the left
	commentstyle=\color{green}, % comment color
	keywordstyle=\color{blue}, % keyword color
	stringstyle=\color{red} % string color
}

% Make sure hyperlinks for table of contents and website links dont have wierd red box
\hypersetup{
	colorlinks,
	citecolor=black,
	filecolor=black,
	linkcolor=black,
	urlcolor=black
}

\begin{document}
	
	% Set up Manual Indenting
	\setlength{\parindent}{15pt}
	\newcommand{\forceindent}{\leavevmode{\parindent=2em\indent}}
	
	
	%----------------------------------------------------------------------------------------
	%	TABLE OF CONTENTS
	%----------------------------------------------------------------------------------------	
	% Create title Page
	\title{ Design Specification Update \\ Sprint 3}
	\author{Team 3 (Java)}
	\date{}
	\maketitle
	
	% Create Table of Contents
	\tableofcontents
	%Set Table of Content depth
	\setcounter{tocdepth}{3}		% Include \subsubsection in ToC
	% Create new Page
	\newpage
	
	%----------------------------------------------------------------------------------------
	%	APPENDICIES
	%----------------------------------------------------------------------------------------
	%Apendix Section
	\appendix
	
		%----------------------------------------------------------------------------------------
		%	APPENDIX A Content - Sample Code/Work Done
		%----------------------------------------------------------------------------------------
		\section{Changes/Updates}
		\forceindent The changes made to the Design Specification was relatively small. These changes are listed below.
		
			%----------------------------------------------------------------------------------------
			%	Frame Preview Bar
			%----------------------------------------------------------------------------------------
			\subsection {Updating the Diagram for Grid Editor}
			\forceindent The grid editor diagram was updated to reflect the changes made to it, namely a color picker was added to the right side of the Grid Editor, the text was removed from the buttons, and a set of translation buttons was added to shift patterns around on the frame. See figure 1 below.
      
      \begin{figure}[ht!]
        \centering
        \includegraphics[width=120mm]{gridEditor_ColorPicker.PNG}
        \caption{Diagram of Grid Editor}
      \end{figure}
    
			%----------------------------------------------------------------------------------------
			%	Frame Preview Bar
			%----------------------------------------------------------------------------------------
			\subsection {Updating the Time-line}
			\forceindent The time-line was edited to reflect the progress made since Sprint 2 and also display goals for the following sprint. The time-line now properly indicates the completion and removal of several objectives. It accurately shows that refinements were made to the Grid Editor and the scrub bar, as well as an indication that there will no longer be a multi-node editor. The time-line now also lists the addition of a progress report presentation.
      
			% Create new Page
			\newpage
		
		%----------------------------------------------------------------------------------------
		%	APPENDIX B Content - Sample Code/Work Done
		%----------------------------------------------------------------------------------------
		\section{Updated Section(s)}
		\forceindent The Single and Multi-Node grid editor sections were removed to reflect the change to using a single Grid Editor. These section(s) are further discussed below.
		
		%----------------------------------------------------------------------------------------
		%	Frame Preview Bar
		%----------------------------------------------------------------------------------------
		\subsection {Grid Editor}
		\forceindent The Single and Multi-Node Editor sections were removed to indicate the change in functionality of the design. This new design is a single Grid Editor with a color picker on the side, where there will eventually be the ability to Shift-Click a second node and a section of nodes will be selected to change the color.
    
    \begin{figure}[ht!]
      \centering
      \includegraphics[width=120mm]{gridEditor.PNG}
      \caption{Diagram of Grid Editor}
    \end{figure}
		% Create new Page
		\newpage
		
		%----------------------------------------------------------------------------------------
		%	APPENDIX C Content - Diagrams for First Development Sprint
		%----------------------------------------------------------------------------------------
		\section{New/Changed Diagrams for Update}
		\forceindent The Following are diagrams were created/updated for relevancy in the updating the Design Specification for the second sprint, these diagrams are presented in the following sections below.
	
			%----------------------------------------------------------------------------------------
			%	CLASS DIAGRAM FOR COLOR, NODE, AND FRAME CLASS
			%----------------------------------------------------------------------------------------
			\subsection {Class Diagram for Grid Editor}
				\forceindent The Class Diagram below in Figure displays the structure of the Grid Editor (so far), in which the Color, Node, and ActionListener classes holds the main mechanism of the Gird Editor, also showing the relationships between these classes. This has been updated from the previous sprint, by removing the Color class (utilizing Java's default Color class) and adding the NodeActionListener class and FrameButtonActionListener class. The diagram is shown in Figure 1 below.
				
				\begin{figure}[ht!]
					\centering
%					\includegraphics[width=120mm]{Class_Diagram_Frame_Node_and_ActionListener_Classes.JPG}
					\caption{Class Diagram of Grid Editor \label{overflow}}
				\end{figure}
			
			%----------------------------------------------------------------------------------------
			%	CLASS DIAGRAM FOR COLOR, NODE, AND FRAME CLASS
			%----------------------------------------------------------------------------------------
			\subsection {Frame Preview Bar Diagram}
			\forceindent The diagram in the figure below displays the structure of the Frame Preview Bar, which contains a list of frames, which have images to show the frame's configuration. There also exists a scroll bar as shown, which allows the user to scroll to show other frames. This diagram was mainly added to present the user with a visual cue to how the Frame Preview Bar should function. The diagram is shown in Figure 2 below.
			
			\begin{figure}[ht!]
				\centering
%				\includegraphics[width=120mm]{potoFramePre.JPG}
				\caption{Diagram of Frame Preview Bar \label{overflow}}
			\end{figure}
			
			%----------------------------------------------------------------------------------------
			%	CLASS DIAGRAM FOR COLOR, NODE, AND FRAME CLASS
			%----------------------------------------------------------------------------------------
			\subsection {Grid Editor Diagram}
			\forceindent The diagram in the figure below displays the structure of the Grid Editor in regards to the second sprint. This was shown to display the progress made on the Grid Editor GUI since the first sprint. The Grid Editor now displays a color when the user enter the RGB values of that color in the node dialog box. This effect was supposed to be focal point of the diagram. The diagram in shown in Figure 3 below.
			
			\begin{figure}[ht!]
				\centering
%				\includegraphics[width=120mm]{protoGrid.JPG}
				\caption{Diagram of Grid Editor \label{overflow}}
			\end{figure}
		
			
		
		% Create new Page (NEED TO USE clearpage because we have pictures that will affect it!)
		\clearpage
		
		
\end{document}